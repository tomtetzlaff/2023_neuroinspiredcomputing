\def\src{../2023_juncatutorials}

\documentclass[8pt,t,usepdftitle=false]{beamer}
\usetheme{Juelich}
\usepackage{setspace}
\usepackage[official]{eurosym}
\usepackage{bm}
\usepackage{mathtools}
%\usepackage{enumitem}\setitemize{itemsep=1ex}
\usepackage[%
backend=bibtex,
style=authoryear,
doi=true,
isbn=true,
url=true,
eprint=false,
sorting=nyt]{biblatex}
\addbibresource{\src/refs.bib}

\fzjset{
  title=regular,
  subtitle=regular,
  part=regular,
}

\setbeamerfont{title}{size*={10pt}{10pt},series=\bfseries}
\setbeamerfont{subtitle}{size*={12pt}{12pt},series=\bfseries\color{white}}
\setbeamerfont{frametitle}{size*={14pt}{14pt},series=\bfseries}
\setbeamertemplate{navigation symbols}{}

\setbeamertemplate{itemize/enumerate body begin}{\normalsize}
\setbeamertemplate{itemize/enumerate subbody begin}{\normalsize}
\setbeamertemplate{itemize/enumerate subsubbody begin}{\normalsize}
\setbeamertemplate{itemize/enumerate subsubsubbody begin}{\normalsize}

\mode<presentation>
{
  \usetheme{default}
  %% \setbeamercovered{transparent}
  \usefonttheme{professionalfonts}
  \usefonttheme{structurebold}
  \usecolortheme[rgb={0,0.3,0.6}]{structure}
}

% Delete this, if you do not want the table of contents to pop up at
% the beginning of each subsection:
\AtBeginSection[]
{
  %\begin{frame}<beamer>
    \begin{frame}[plain]
    \frametitle{Outline}
    \tableofcontents[currentsection]
  \end{frame}
}

\setlength{\leftmarginii}{3ex}

\setbeamercolor{alerted text}{fg=fzjblue}

\renewcommand{\arraystretch}{1.5}

%%%%%%%%%%%%%%%%%%%%%%%%%%%%%%%%%%%%%%%%%%%%%%%%%%%%%%%%%%%%%%%%%%%%%%%%%%%%%%%%%%%%%%%%%%%%%%%%%
%% macros
\def\figpath{\src/figures}

\hypersetup{
  pdftitle={Anatomy of the neocortex},
  pdfauthor={Tom Tetzlaff}
}

%%%%%%%%%%%%%%%%%%%%%%%%%%%%%%%%%%%%%%%%%%%%%%%%%%%%%%%%%%%%%%%%%%%%%%%%%%%%%%%%%% 
\title{%
  {\ \\\LARGE\bf Anatomy of the neocortex}\\[1ex]
}
\subtitle{%
  {\normalsize\mdseries Tom Tetzlaff}%
  {\hfill\tiny\url{t.tetzlaff{at}fz-juelich.de}}\\
  {\footnotesize\mdseries Institute of Neuroscience and Medicine (INM-6), J\"ulich Research Centre and JARA}
  {\hfill\tiny\url{http://www.csn.fz-juelich.de}}
  \\
  {\tiny\mdseries Neuroinspired Computing seminar, J\"ulich, 19.10.2023}
}
\date{}
\author{}
\institute{}

%%%%%%%%%%%%%%%%%%%%%%%%%%%%%%%%%%%%%%%%%%%%%%%%%%%%%%%%%%%%%%%%%%%%%%%%%%%%%%%%%%%%%%%%%%%%%%%%%
\begin{document}
\maketitle
%%%%%%%%%%%%%%%%%%%%%%%%%%%%%%%%%%%%%%%%%%%%%%%%%%%%%%%%%%%%%%%%%%%%%%%%%%%%%%%%%%%%%%%%%%%%%%%%%
\begin{frame}[plain]
  \begin{center}
    \parbox{0.9\linewidth}{
      \vspace{0.95\textheight}
      \parbox[c]{0.1\linewidth}{%
        \href{https://creativecommons.org/licenses/by-sa/4.0}{%
          \includegraphics[width=\linewidth]{\figpath/by-sa.png}}}
      \parbox[c]{0.9\linewidth}{\scriptsize%
        ~~{}This presentation is provided under the terms of the Creative Commons Attribution-ShareAlike License 4.0.
      }
    }    
  \end{center}
\end{frame}
%%%%%%%%%%%%%%%%%%%%%%%%%%%%%%%%%%%%%%%%%%%%%%%%%%%%%%%%%%%%%%%%%%%%%%%%%%%%%%%%%%%%%%%%%%%%%%%%%
\def\ttl{Outline}
\pdfbookmark[2]{Outline}{Outline}
\begin{frame}[plain]
  \frametitle{\ttl}
  \tableofcontents
\end{frame}
%%%%%%%%%%%%%%%%%%%%%%%%%%%%%%%%%%%%%%%%%%%%%%%%%%%%%%%%%%%%%%%%%%%%%%%%%%%%%%%%%%%%%%%%%%%%%%%%%
\def\ttl{Macroscopic cortex structure}
\section{\ttl}
\begin{frame}[plain,t]
  \frametitle{\ttl}
  \framesubtitle{Classification}
  \begin{columns}
    \begin{column}{0.5\linewidth}
      \begin{itemize}
        \itemsep1ex
      \item forebrain (anterior part of the brain): 
        \begin{itemize}\itemsep1ex
          % \item relatively undeveloped in lower vertebrates 
          %   (fish, amphibia, reptiles)
          % \item largest part of the mammalian brain 
        \item diencephalon, e.g.~thalamus and hypothalamus
        \item telencephalon (cerebrum)
          \begin{itemize}
          \item basal ganglia
            (motor control, cognition, emotions, learning)
          \item limbic system, e.g.~amygdala (emotions)
          \item olfactory bulb (smell)
          \item cerebral cortex\\ (archeocortex + \alert{neocortex})
          \end{itemize}
        \end{itemize}
      \end{itemize}      
    \end{column}
    \begin{column}{0.5\linewidth}
      \begin{center}
        \vspace*{-1cm}
      \parbox{0.8\linewidth}{
        \includegraphics[width=\linewidth]{\figpath/EmbryonicBrain.png}\\
        \hspace*{\fill}\tiny\color{gray}{(Embryonic vertebrate brain; Wikipedia)}
      }\\[1ex]
      \parbox{0.9\linewidth}{
        \includegraphics[width=\linewidth]{\figpath/brainformation_bioninja.jpg}\\
        \hspace*{\fill}\tiny\color{gray}{(\url{ib.bioninja.com.au})}
      }%                    
      \end{center}
    \end{column}
  \end{columns}
\end{frame}
%%%%%%%%%%%%%%%%%%%%%%%%%%%%%%%%%%%%%%%%%%%
\begin{frame}[plain]
  \frametitle{\ttl}
  \framesubtitle{Cortex folds}
  \begin{columns}
    \begin{column}{0.5\linewidth}
      \begin{itemize}
        \itemsep1ex
      \item<1-> neocortex = surface of the mammalian brain (gray matter)
      \item<1-> strongly folded in 'higher' mammals
        \begin{itemize}\itemsep1ex
        \item large surface (relatively to skull volume)
        \item efficient wiring between cortical areas through white matter
        \end{itemize}
      \item<2-> folds = 'sulci'
      \item<2-> region between adjacent folds = 'gyrus'
      \item<2-> location of (deep) sulci is consistent across individuals of the same species    
      \end{itemize}
    \end{column}
    \begin{column}{0.5\linewidth}
      \vspace*{-1.5cm}
      \begin{center}
        \only<1->{
          \parbox[b]{0.7\linewidth}{
            \includegraphics[width=\linewidth]{\figpath/Gray717.png}
            \hspace*{\fill}{\tiny\color{gray}\parencite{Gray18}}
          }\\
        }
        \only<2->{
          \parbox[b]{\linewidth}{
            \includegraphics[width=\linewidth]{\figpath/crtcncs_fig1_1_2.png}
            \hspace*{\fill}{\tiny\color{gray}\parencite{Abeles91}}     
          }
        }
      \end{center}
    \end{column}    
  \end{columns}
\end{frame}
%%%%%%%%%%%%%%%%%%%%%%%%%%%%%%%%%%%%%%%%%%%
\begin{frame}[plain]
  \frametitle{\ttl}
  \framesubtitle{Cortex thickness}
  \begin{columns}
    \begin{column}{0.45\linewidth}
      \begin{itemize}\itemsep1ex
      \item<1-> thickness of human cortex:\\
        $1$--$4.5\,\text{mm}$ (average: $2.5\,\text{mm}$)
      \item<2-> Evolution increased mainly the surface of the cortex.
      \item<2-> Its vertical organisation
      (thickness, cortex layers, cell proportions, etc.) remained relatively constant.
      \end{itemize}
    \end{column}
    \begin{column}{0.55\linewidth}
      \begin{center}
        \vspace*{-0.5cm}
        \parbox{\linewidth}{
          \includegraphics[width=\linewidth]{\figpath/crtcncs_fig1_1_3.png}
          \hspace*{\fill}\tiny\color{gray}{(numbers represent relative brain mass; \cite{Abeles91})}
        }
        \end{center}
    \end{column}
  \end{columns}
\end{frame}
%%%%%%%%%%%%%%%%%%%%%%%%%%%%%%%%%%%%%%%%%%%
\begin{frame}[plain]
  \frametitle{\ttl}
  \framesubtitle{Cerebral hemispheres}
  \begin{itemize}\itemsep1ex
  \item two mirror-symmetric cerebral hemispheres
  \item interconnected via the Corpus callosum (white matter)
  \end{itemize}
  \begin{center}
    \parbox{\linewidth}{
      \parbox[b]{0.45\linewidth}{
        \includegraphics[width=\linewidth]{\figpath/Gray717.png}
        \hspace*{\fill}\tiny\color{gray}{(Gray, 1918)}
      }%
      \hfill%
      \parbox[b]{0.5\linewidth}{
        \includegraphics[width=\linewidth]{\figpath/humbrain.png}
        \hspace*{\fill}\tiny\color{gray}{\parencite{Abeles91}}
      }      
    }
  \end{center}
\end{frame}
%%%%%%%%%%%%%%%%%%%%%%%%%%%%%%%%%%%%%%%%%%%
\begin{frame}[plain]
  \frametitle{\ttl}
  \framesubtitle{Cerebral hemispheres}
  \begin{columns}
    \begin{column}{0.4\linewidth}
      \begin{itemize}\itemsep1ex
      \item brain function is sometimes lateralised (preference for one hemisphere)
      \item in general, however, involvement of both hemispheres
      \item<2-> example: visual pathway
        \begin{itemize}\itemsep1ex
        \item each hemisphere receives information from both eyes
        \item however, preference of input from left part of the visual 
        field for right hemisphere (and vice versa)
        \end{itemize}
      \end{itemize}      
    \end{column}
    \begin{column}{0.6\linewidth}
      \begin{center}
        \vspace*{-0.5cm}
        \parbox{\linewidth}{
          \includegraphics[width=\linewidth]{\figpath/lateralisation.png}
          %\hspace*{\fill}\tiny\color{gray}{()}
        }\\[1ex]
        \only<2->{
        \parbox{0.6\linewidth}{
          \includegraphics[width=\linewidth]{\figpath/visualpath.jpg}
          %\hspace*{\fill}\tiny\color{gray}{()}
        }}
      \end{center}
    \end{column}
  \end{columns}
\end{frame}
%%%%%%%%%%%%%%%%%%%%%%%%%%%%%%%%%%%%%%%%%%%
\begin{frame}[plain]
  \frametitle{\ttl}
  \framesubtitle{Cortical areas}
  \begin{columns}
    \begin{column}{0.5\linewidth}
      \vspace*{-3ex}
      \begin{itemize}\itemsep1ex
      \item subdivision of cortex into areas according to
        \begin{itemize}\itemsep1ex
        \item \emph{functional} properties
        \item \emph{histo-anatomical} features (cytoarchitecture, e.g.~Brodman areas)
        \end{itemize}
      \item frequent coincidence of both definitions,
        e.g.,\\ primary visual cortex (V1) = Brodman area 17
      \item<2-> further segmentation of cortical areas into functional subdomains
      \item<2->[] examples:
        \begin{itemize}\itemsep1ex
        \item somatosensory cortex
        \item rat ``barrel'' cortex
        \end{itemize}
      \end{itemize}
      \vspace{1ex}
    \end{column}
    \begin{column}{0.5\linewidth}
      \begin{center}
        \only<1>{
          \vspace*{-0.5cm}
          \parbox{\linewidth}{
            \includegraphics[width=\linewidth]{\figpath/functorganisation2.png}
            % \hspace*{\fill}\tiny\color{gray}{()}     
          }\\[2ex]
          \parbox{0.8\linewidth}{
            \includegraphics[width=\linewidth]{\figpath/crtcncs_fig1_3_5.png}
            \hspace*{\fill}\tiny\color{gray}{\parencite{Abeles91}}
          }
        }
        \only<2>{
          \vspace*{-0.7cm}
          \parbox{0.6\linewidth}{
            \includegraphics[width=\linewidth]{\figpath/homunculus.jpg}
            \hspace*{\fill}\tiny\color{gray}{(somatosensory cortex)}      %% unknown figure source
          }
        }
        \only<3>{
          \vspace*{-1.4cm}
          \parbox{\linewidth}{
            \hspace{0.5cm}
            \includegraphics[width=0.75\linewidth]{\figpath/Vasconcelos11_15408_fig5.png}\\[-3ex]
            \hspace*{\fill}{\tiny\color{gray}\parencite{Vasconcelos11_15408}}
          }
        }
      \end{center}      
    \end{column}
  \end{columns}
  \only<2>{
    \parbox{\linewidth}{
      \parbox{0.6\linewidth}{
        \includegraphics[width=\linewidth]{\figpath/barrelcortexA.png}
      }\hfill
      \parbox{0.35\linewidth}{
        \includegraphics[width=\linewidth]{\figpath/barrelcortexB.png}
      }\\[1ex]
      \hspace*{\fill}\tiny\color{gray}{(barrel cortex)}    %% unknown figure source
    }
  }
  \only<3>{
    \fbox{\parbox{\linewidth}{
      \emph{however:}
      \begin{itemize}
      \item \emph{The fact that certain regions in the cortex are specialized in some modality or function does not imply that the rest of the cortex is not involved in this function.}
      \item \emph{The representation of each function is typically widely distributed across the cortex and involves many areas.}
      \item \emph{This is true for all spatial scales -- from entire cortical areas to individual neurons\\ (cf., ``receptive fields'').}
      \end{itemize}
    }}
  }  
\end{frame}
%%%%%%%%%%%%%%%%%%%%%%%%%%%%%%%%%%%%%%%%%%%%%%%%%%%%%%%%%%%%%%%%%%%%%%%%%%%%%%%%%%%%%%%%%%%%%%%%%
\def\ttl{Microscopic cortex structure}
\section{\ttl}
\begin{frame}[plain]
  \frametitle{\ttl}
  \framesubtitle{Cell types}
  \begin{itemize}\itemsep1ex
  \item cortical tissue mainly composed of two cell types:
    \begin{itemize}\itemsep1ex
    \item \emph{neuroglia (glia)}:
      \begin{itemize}\itemsep1ex
      \item important role in development of the brain
      \item metabolic supportive role
      \item control ionic composition of extracellular space
      \item form myelin sheath around axons of neurons
      \item do not take part in interaction between neurons on a millisecond scale
      \item however: play a role in slow modulations
      \end{itemize}
    \item \emph{nerve cells (neurons)}:
      \begin{itemize}\itemsep1ex
      \item carry out information processing and storage in the brain
      \end{itemize}
    \end{itemize}
  \item neuron-glia ratio $\approx$ 1:1 
  \end{itemize}
\end{frame}
%%%%%%%%%%%%%%%%%%%%%%%%%%%%%%%%%%%%%%%%%%%
\begin{frame}[plain]
  \frametitle{\ttl}
  \framesubtitle{Cortical neurons}
  \begin{center}
  \parbox{0.85\linewidth}{
    \includegraphics[width=\linewidth]{\figpath/Complete_neuron_cell_diagram.png}
    \hspace*{\fill}\tiny\color{gray}{(Wikipedia)}
  }    
  \end{center}
\end{frame}
\begin{frame}[plain]
  \frametitle{\ttl}
  \framesubtitle{Cortical neurons}
  \begin{center}
    \vspace*{-2ex}
  \parbox{0.8\linewidth}{
    \includegraphics[width=\linewidth]{\figpath/Lee06_e29_fig6f.png}\\[1ex]
    \hspace*{\fill}\tiny\color{gray}{(pyramidal neuron in mouse visual cortex; \cite{Lee06_e29})}
  }    
  \end{center}
\end{frame}
%%%%%%%%%%%%%%%%%%%%%%%%%%%%%%%%%%%%%%%%%%%
\begin{frame}[plain]
  \frametitle{\ttl}
  \framesubtitle{Neuron types}
  \begin{columns}
    \begin{column}{0.5\linewidth}
      \vspace*{-3ex}
       \begin{itemize} 
       \item<1-> classification of different neuron types according to 
         \begin{itemize}
         \item morphology of cell body and dendritic tree, e.g.
           \begin{itemize}
           \item pyramidal cells: triangular soma profile, 
             long apical dendrites extending from cell body towards 
             the cortex surface, basal dendrites close to soma
           \item stellate cells: star shaped, dendrites extending in all directions
           \end{itemize}
         \item<2-> type of synapse, e.g.
           \begin{itemize}
           \item smooth cells
           \item  spiny cells
           \end{itemize}
         \item<3-> location of the cell body (layer)
         \item<4-> other criteria 
           (e.g.,~brain region to which axons project, neurotransmitters, 
           other biochemical aspects)
         \end{itemize}
       \item<5-> classification by \textcite{Abeles91}:
         \begin{itemize}
         \item \alert{pyramidal cell}
         \item \alert{spiny stellate cell}
         \item \alert{smooth stellate cell}
         \end{itemize}
       \end{itemize}
     \end{column}
    \begin{column}{0.5\linewidth}
      \begin{center}
        \vspace*{-5ex}
        \only<1->{
          \parbox[t]{0.45\linewidth}{
            {\small Pyramidal cell}\\
            \includegraphics[width=\linewidth]{\figpath/crtcncs_fig1_2_2.png}
            % \hspace*{\fill}{\tiny\color{gray}\parencite{Abeles91}}
          }\hfill
          \parbox[t]{0.45\linewidth}{
            {\small Stellate cell}\\          
            \includegraphics[width=\linewidth]{\figpath/crtcncs_fig1_2_3.png}\\
            \hspace*{\fill}{\tiny\color{gray}\parencite{Abeles91}}
          }\\[2ex]
        }
        \only<2->{
          \parbox[t]{0.7\linewidth}{
            {\small Dendritic spines}\\
            \includegraphics[width=0.65\linewidth]{\figpath/crtcncs_fig1_2_1.png}
            \hspace*{\fill}{\tiny\color{gray}\parencite{Abeles91}}
          }
        }
  \end{center}
    \end{column}
  \end{columns}
\end{frame}
\begin{frame}[plain]
  \frametitle{\ttl}
  \framesubtitle{Neuron types}
  \begin{columns}
    \begin{column}{0.5\linewidth}
      \vspace*{-2ex}
      \begin{itemize}
      \item<1-> \emph{pyramidal cell}:
      \begin{itemize}
      \item dendrites covered with spines
      \item axon leaves cortex into white matter but has numerous branches close cell body
      \item axons make \emph{excitatory} synapses
      \item cell receives input from inhibitory cells mainly at the soma and from excitatory cells at the basal and apical dendrite
      \end{itemize}
      \item<2-> \emph{spiny stellate cell}:
      \begin{itemize}
      \item axon branches out within the cortex (rarely leaves the cortex)
      \item axons make \emph{excitatory} synapses
      \item soma receives exclusively inhibitory and dendrites mostly excitatory inputs
      \end{itemize}
      \item<3-> \emph{smooth stellate cell} (e.g., basket cell):
      \begin{itemize}
      \item axon branches out only in the cortex
      \item axons make \emph{inhibitory} synapses (GABA)
      \item both the soma and the dendrites receive a mixture of excitatory and inhibitory inputs
      \end{itemize}
    \end{itemize}           
    \end{column}
    \begin{column}{0.5\linewidth}
      \begin{center}
        \vspace*{-0.5cm}
        \only<1>{
          \parbox{0.8\linewidth}{
            \includegraphics[width=\linewidth]{\figpath/crtcncs_fig1_2_2.png}\\
            \hspace*{\fill}{\tiny\color{gray}(pyramidal cell; \cite{Abeles91})}
          }
        }
        \only<2>{
          \parbox{0.9\linewidth}{
            \includegraphics[width=\linewidth]{\figpath/crtcncs_fig1_2_3.png}\\
            \hspace*{\fill}{\tiny\color{gray}(spiny stellate cell; \cite{Abeles91})}
          }        
        }
        \only<3>{
          \parbox{0.9\linewidth}{
            \includegraphics[width=\linewidth]{\figpath/crtcncs_fig1_2_4.png}\\
            \hspace*{\fill}{\tiny\color{gray}(smooth stellate cell; \cite{Abeles91})}
          }        
        }
      \end{center}
    \end{column}
  \end{columns}
\end{frame}
\begin{frame}[plain]
  \frametitle{\ttl}
  \framesubtitle{Neuron types}
  \parbox{\linewidth}{
    \includegraphics[width=\linewidth]{\figpath/Binzegger04_8441_fig2.png}\\
    \hspace*{\fill}{\tiny\color{gray}\parencite{Binzegger04}}
  }        
\end{frame}
%%%%%%%%%%%%%%%%%%%%%%%%%%%%%%%%%%%%%%%%%%% 
\begin{frame}[plain]
  \frametitle{\ttl}
  \framesubtitle{Cortical layers}
  \vspace*{-1ex}
  \begin{itemize}
  \item subdivision of neocortex into 6 (or 5) layers
    \begin{itemize}
    \item \alert{layer I}: 
      very few neurons, only dendrites and mesh of horizontal axons
    \item \alert{layer II/III}: 
      small pyramidal cells
    \item \alert{layer IV}: 
      small and medium-size pyramidal cells and stellate cells 
      in layer IVA, 
      almost exclusively stellate cells in IVB
    \item \alert{layer V}: 
      all cell types, large pyramidal cells dominate
    \item \alert{layer VI}: 
      small and medium-size pyramidal cells
    \end{itemize}
  \end{itemize}
  \vspace*{1ex}
  \parbox{\linewidth}{
    \parbox{0.38\linewidth}{
      \includegraphics[width=\linewidth]{\figpath/layers.png}\\
      % \hspace*{\fill}{\tiny\color{gray}\parencite{}}
    }%
    \hfill    
    \parbox{0.55\linewidth}{
      \vspace*{-5ex}
      \includegraphics[width=\linewidth]{\figpath/crtcncs_fig1_3_4.png}\\
      {\small primary motor area}%
      \hfill
      {\small primary sensory area}\\
      \hspace*{\fill}{\tiny\color{gray}\parencite{Abeles91}}
    }
  }        
\end{frame}
%%%%%%%%%%%%%%%%%%%%%%%%%%%%%%%%%%%%%%%%%%%
\begin{frame}[plain]
  \frametitle{\ttl}
  \framesubtitle{Neuron density}
  \only<1>{
    \begin{center}
      \parbox{\linewidth}{       
        \parbox{0.4\linewidth}{
          \includegraphics[width=\linewidth]{\figpath/layers}
        }%
        \hfill
        \parbox{0.6\linewidth}{
          \vspace*{-0.7cm}
          \includegraphics[width=\linewidth]{\figpath/pyram}
        }\\
        \hspace*{\fill}{\tiny\color{gray}\parencite{Abeles91}}
      }
    \end{center}
  }
  \only<2->{
    \begin{center}
      \vspace*{-0.35cm}
      \parbox{\linewidth}{       
        \parbox{0.5\linewidth}{
          \includegraphics[width=\linewidth]{\figpath/axons}
        }%
        \hfill
        \parbox{0.5\linewidth}{
          \includegraphics[width=\linewidth]{\figpath/Abeles91_tables_1_5_1_and_1_5_4.png}
        }\\
        \hspace*{\fill}{\tiny\color{gray}\parencite{Abeles91}}
      }
    \end{center}
  }         
\end{frame}
%%%%%%%%%%%%%%%%%%%%%%%%%%%%%%%%%%%%%%%%%%%%%%%%%%%%%%%%%%%%%%%%%%%%%%%%%%%%%%%%%%%%%%%%%%%%%%%%%
\def\ttl{Cortical connectivity}
\section{\ttl}
\begin{frame}[plain]
  \frametitle{\ttl}
  \framesubtitle{Synapses}
  \begin{itemize}
  \item communication between nerve cells via synapses
    \begin{itemize}
    \item<1-> \emph{electrical synapses} (gap junctions): 
      \begin{itemize}
      \item direct electrical contact between two cells through ion channels 
        (membrane proteins) spanning membranes of both cells
        of both cells
      \item in the cortex only between GABAergic interneuons
      \end{itemize}
    \item<2-> \emph{chemical synapses}:
      \begin{itemize}
      \item signaling between pre- and postsynaptic neuron via neurotransmitters 
        diffusing across the synaptic cleft
      \end{itemize}
    \end{itemize}
  \end{itemize}
  \parbox{\linewidth}{
    \only<1->{
      \parbox{0.48\linewidth}{
        \includegraphics[width=\linewidth]{\figpath/Gap_cell_junction.png}\\
        \hspace*{\fill}{\tiny\color{gray}(gap junction; Wikipedia)}
      }
    }
    \only<2->{
      \parbox{0.48\linewidth}{
        \includegraphics[width=\linewidth]{\figpath/synapse2.jpg}\\ %% unknown source
        \hspace*{\fill}{\tiny\color{gray}(chemical synapse)}        
      }
    }
  }  
\end{frame}
\begin{frame}[plain]
  \frametitle{\ttl}
  \framesubtitle{Chemical synapses}  
  \begin{center}
    \vspace*{-0.3cm}
  \parbox{0.62\linewidth}{
    \includegraphics[width=\linewidth]{\figpath/synapse.jpg}\\ %% unknown source
  }    
  \end{center}
\end{frame}
\begin{frame}[plain]
  \frametitle{\ttl}
  \framesubtitle{Chemical synapses}
  \begin{itemize}\itemsep1ex
  \item<1-> chemical synapses are slower than electrical synapses ($1$\ldots$10\,\text{ms}$)
  \item<1-> modulation of synaptic strength through
    \begin{itemize}\itemsep1ex
    \item amount of neurotransmitter
    \item (in)activation of postsynaptic neurotransmitter receptors
    \item postsynaptic ion-channel density
    \end{itemize}
    $\curvearrowright$ \emph{learning}
  \item<2-> postsynaptic effect depends on neurotransmitter, receptor and 
    ion-channel type, e.g.
    \begin{itemize}\itemsep1ex
    \item binding of Glutamate at NMDA receptors\\
      $\curvearrowright$ opening of a nonselective cation channel
      (Na$^+$, K$^+$, Ca$^{2+}$)\\
      $\curvearrowright$ Na$^+$ (and Ca$^{2+}$) influx, K$^+$ outflow\\
      $\curvearrowright$ depolarisation of postsynaptic membrane, 
      i.e., \emph{excitation}
    \item binding of GABA (Gamma-aminobutyric acid) at GABA receptors\\
      $\curvearrowright$ opening of Cl$^-$ channels in postsynaptic membrane\\
      $\curvearrowright$ Cl$^-$ influx\\
      $\curvearrowright$ hyperpolarisation, i.e., \emph{inhibition}
    \end{itemize}
  \item<3->[]but also on postsynaptic membrane potential (reversal potentials) 
    and ion concentrations
  \end{itemize}
\end{frame}
%%%%%%%%%%%%%%%%%%%%%%%%%%%%%%%%%%%%%%%%%%% 
\begin{frame}[t,plain]
  \frametitle{\ttl}
  \framesubtitle{Afferent (incoming) connections}
  \begin{columns}
    \begin{column}{0.5\linewidth}
      \begin{itemize}\itemsep1ex
      \item external (non-local) inputs from
        \begin{itemize}\itemsep1ex
        \item brain stem,
        \item subcortical nuclei
        \item in particular, thalamus (thalamo-cortical afferents):
          mainly to layer IV
        \item other cortical areas (cortico-cortical afferents)
          via white matter:
          mostly to superficial layers
        \end{itemize}
      \item inputs from cortical neurons in the local neighborhood
      \end{itemize}
    \end{column}
    \begin{column}{0.5\linewidth}
      \begin{center}
        \vspace*{-1cm}
      \parbox{\linewidth}{
        \includegraphics[width=\linewidth]{\figpath/crtcncs_fig1_3_7.png}\\
        \parbox{\linewidth}{\tiny\color{gray}(a,b: specific thalamo-cortical afferents;
          c,d: non-specific afferents;
          e,f: cortico-cortical afferents; \cite{Abeles91})
        }        
      }        
      \end{center}
    \end{column}
  \end{columns}
\end{frame}
%%%%%%%%%%%%%%%%%%%%%%%%%%%%%%%%%%%%%%%%%%%
\begin{frame}[plain]
  \frametitle{\ttl}
  \framesubtitle{Efferent (outgoing) connections}
  \begin{itemize}\itemsep1ex
  \item almost all pyramidal and some spiny stellate cells send axons 
    into the white matter, in particular
    \begin{itemize}\itemsep1ex
    \item cortico-cortico efferents from layer II/III pyramidal cells
    \item cortico-thalamic efferents from layer VI pyramidal cells
    \item axons to brain stem and spinal cord from large layer V pyramidal cells
    \end{itemize}
  \end{itemize}  
\end{frame}
%%%%%%%%%%%%%%%%%%%%%%%%%%%%%%%%%%%%%%%%%%%
\begin{frame}[plain]
  \frametitle{\ttl}
  \framesubtitle{Vertical (laminar) connectivity}
  \vspace*{-2ex}
  \begin{itemize}
  \item layer- and cell-type specific local connections
  \end{itemize}
  \begin{center}
    \only<1>{
      \parbox{0.75\linewidth}{
        \includegraphics[width=\linewidth]{\figpath/Thomson03_5_fig1.png}\\
        \hspace*{\fill}{\tiny\color{gray}\parencite{Thomson03}}
      }
    }
    \only<2>{      
      \parbox{0.9\linewidth}{
        \includegraphics[width=\linewidth]{\figpath/Binzegger04_8441_fig12_mod.png}\\
        \hspace*{\fill}{\tiny\color{gray}\parencite{Binzegger04}}
      }
    }
  \end{center}
\end{frame} 
%%%%%%%%%%%%%%%%%%%%%%%%%%%%%%%%%%%%%%%%%%%
\begin{frame}[plain]
  \frametitle{\ttl}
  \framesubtitle{Horizontal connectivity}
  \begin{itemize}
  \item local synapses established by local axon collaterals arborizing within 
    $\sim{}0.5\,\text{mm}$ (all neuron types)
  \item intrinsic horizontal long-range connections of pyramidal cells 
    over distances up to several millimeter (within gray matter)
  \item extrinsic long-range connections of pyramidal cells through white matter
  \end{itemize}
  \begin{center}
    \parbox{\linewidth}{
      \includegraphics[width=\linewidth]{\figpath/Voges07_PHD_fig2}\\
      \hspace*{\fill}{\tiny\color{gray}\parencite{Voges10_137}}
    }            
  \end{center}
\end{frame}
%%%%%%%%%%%%%%%%%%%%%%%%%%%%%%%%%%%%%%%%%%%
\begin{frame}[plain]
  \frametitle{\ttl}
  \framesubtitle{Local connectivity}
  \begin{itemize}
  \item probability of synaptic connection between adjacent
    cortical neurons decays to zero within a horizontal distance of $\sim{}0.5\,\text{mm}$
  \end{itemize}
  \begin{center}
    \parbox{0.7\linewidth}{
      \includegraphics[width=\linewidth]{\figpath/Hellwig00_fig7}\\
      \hspace*{\fill}{\tiny\color{gray}\parencite{Hellwig00}}
    }      
  \end{center}
\end{frame}
%%%%%%%%%%%%%%%%%%%%%%%%%%%%%%%%%%%%%%%%%%%
\begin{frame}[plain]
  \frametitle{\ttl}
  \framesubtitle{Local connectivity}
  %\vspace{-0.5cm}
  \begin{itemize}
  \item detailed local connectivity data from new methods
  \end{itemize}
  \vspace{2ex}
  \parbox{\linewidth}{
    \parbox[t]{0.55\linewidth}{
      \ \\
      \includegraphics[width=\linewidth]{\figpath/Motta19_eaay3134_fig1.jpg}
    }%
    \hfill
    \parbox[t]{0.44\linewidth}{
      \ \\
      \includegraphics[width=\linewidth]{\figpath/Motta19_eaay3134_fig3.jpg}
    }\\        
    \hspace*{\fill}{\tiny\color{gray}\parencite{Motta19_eaay3134}}
      }
\end{frame}
%%%%%%%%%%%%%%%%%%%%%%%%%%%%%%%%%%%%%%%%%%%
\begin{frame}[plain]
  \frametitle{\ttl}
  \framesubtitle{Intrinsic long-range connectivity}
  \vspace*{-2ex}
  \begin{columns}
    \begin{column}{0.5\linewidth}
      \begin{itemize}
      \item 'patchy' projection patterns:
        pyramidal cells project to distant (few mm) clusters
        of target cells
      \end{itemize}
    \end{column}
    \begin{column}{0.5\linewidth}
      \only<1>{
        \begin{center}
          \vspace*{-2ex}
          \parbox[t]{\linewidth}{
            \includegraphics[width=\linewidth]{\figpath/Lund03_fig3B_mod}\\
            \hspace*{\fill}{\tiny\color{gray}\parencite{Lund03}}
          }
        \end{center}
      } 
    \end{column}
  \end{columns}
  \only<2>{
    \begin{center}
      \parbox[t]{\linewidth}{
        \includegraphics[width=\linewidth]{\figpath/Voges07_PHD_fig1}\\
        \hspace*{\fill}{\tiny\color{gray}\parencite{Voges10_137}}
      }
    \end{center}
  }
\end{frame}
%%%%%%%%%%%%%%%%%%%%%%%%%%%%%%%%%%%%%%%%%%%
\begin{frame}[plain]
  \frametitle{\ttl}
  \framesubtitle{Extrinsic long-range connectivity}
  \vspace*{-2ex}
  \begin{columns}
    \begin{column}{0.6\linewidth}
      \begin{itemize}\itemsep1ex
      \item<1-> white-matter connections between cortical areas
      \item<2-> highly inhomogeneous and area specific
      \item<3-> hierarchical organization with layer-specific \\feedforward (bottom-up) and feedback (top-down) connections
      \end{itemize}      
    \end{column}
    \begin{column}{0.4\linewidth}
      \only<3>{
        \begin{center}
          \parbox{\linewidth}{\centering
            % \includegraphics[width=0.8\linewidth]{\figpath/Markov14_225_fig5.png}\\
            % \hspace*{\fill}{\tiny\color{gray}\parencite{Markov14}}
            \includegraphics[width=\linewidth]{\figpath/Markov13_187.jpg}\\
            \hspace*{\fill}{\tiny\color{gray}\parencite{Markov13_187}}
      }        
    \end{center}
  }      
    \end{column}
  \end{columns}
  \only<1>{
    \begin{center}
      \vspace*{-1cm}
      \parbox[t]{\linewidth}{
        \includegraphics[width=\linewidth]{\figpath/Schmidt18_1409_fig1.png}\\
        \hspace*{\fill}{\tiny\color{gray}\parencite{Schmidt18_1409}}
      }
    \end{center}
  }
  \only<2>{
    \begin{center}
      \parbox{\linewidth}{\centering
        \includegraphics[width=\linewidth]{\figpath/Budd12_fig5_a_c.png}\\
        \hspace*{\fill}{\tiny\color{gray}\parencite{Budd12_42}}
      }        
    \end{center}
  }
\end{frame}
%%%%%%%%%%%%%%%%%%%%%%%%%%%%%%%%%%%%%%%%%%%%%%%%%%%%%%%%%%%%%%%%%%%%%%%%%%%%%%%%%%%%%%%%%%%%%%%%%
\def\ttl{Universality of neocortical architecture}
\section{\ttl}
\begin{frame}[t,plain]
  % 
  \frametitle{\ttl}
  %% 
  \only<2>{
    %\vspace*{1ex}
    Cytoarchitecture of\\
    different human cortices:\\[-2ex] 
    \parbox{\linewidth}{
      \hspace*{\fill}\includegraphics[width=0.7\linewidth]{\figpath/Yanez05_344_fig2_labeled.png}\\
      \hspace*{\fill}{\tiny\color{gray}\parencite{Yanez05_344,}}      
    }
    \\[-17ex]
    \parbox{0.25\linewidth}{
      \includegraphics[width=\linewidth]{\figpath/Gray727-Brodman.png}\\[-2ex]
      \hspace*{\fill}\tiny\color{gray} (Wikipedia)
    }
  }
  %%% 
  \only<3>{
    %\vspace*{1ex}
    Cytoarchitecture of cortices\\
    from different mammals:\\[-4ex]
    \parbox{\linewidth}{
      \hspace*{\fill}\includegraphics[width=0.55\linewidth]{\figpath/DeFelipe11_fig8.jpg}\\
      \hspace*{\fill}{\tiny\color{gray}\parencite{DeFelipe11_29}}      
    }
    \\[-35ex]
    \parbox{0.25\linewidth}{
      \includegraphics[width=\linewidth]{\figpath/Lobes_of_the_brain_NL.pdf}\\
      \hspace*{\fill}\tiny\color{gray} (adapted from Wikipedia)
    }
  }
  %%% 
  \begin{itemize}
  \item<1,4-> structure of neocortex is variable across functional areas, individuals \& species\\
    (thickness, vertical organization, cell density, connection density)
  \item<1,4-> but: \emph{similarities are more striking than differences}
    \only<1>{\\[4ex]
      \parbox{\linewidth}{\centering
        \includegraphics[width=0.8\linewidth]{\figpath/DeFelipe11_fig10.jpg}\\
        \hspace*{\fill}{\tiny\color{gray}\parencite{DeFelipe11_29}}
      }}    
  \item<4->[$\curvearrowright$] \emph{universal computational device?} {\tiny\color{gray}(Mountcastle, 1978)}\\[1ex]
  \item<5-> \emph{\Large What ``universal'' algorithm is it implementing?}\\[1ex]
  \item<6-> functional specificity may arise from specific connectivity between
    \begin{itemize}
    \item subcortical brain regions (e.g.~thalamus)
      and different cortex areas
    \item different cortex areas
    \end{itemize}
    \visible<6>{
      \ \\[1ex]
      \parbox{\linewidth}{\centering
        \includegraphics[width=0.8\linewidth]{\figpath/Budd12_fig5_a_c.png}\\
        \hspace*{\fill}{\tiny\color{gray}\parencite{Budd12_42}}
      }    
    }
  \end{itemize}
\end{frame}
%%%%%%%%%%%%%%%%%%%%%%%%%%%%%%%%%%%%%%%%%%%%%%%%%%%%%%%%%%%%%%%%%%%%%%%%%%%%%%%%%%%%%%%%%%%%%%%%%
\def\ttl{Properties of cortical networks -- an overview}
\section{\ttl}
\begin{frame}[plain]
  \frametitle{\ttl}
  \framesubtitle{}
    \begin{itemize}
    \itemsep1ex
  \item<1-> large (but not infinite) network size:
    $\sim{}10^{4,\ldots{},5}$ neurons/mm$^3$,
    $\sim{}10^{10}$ neurons in human cortex
  \item<2-> large diversity of cell types (morphology + electrophysiology)
  \item<3-> classification of neurons into ``excitatory'' and ``inhibitory'' (Dale's law)
  \item<4-> large fan-in/out: $\sim{}10^2,\ldots, 10^4$ synapses per neuron
  \item<5-> short- and long-range connections:
    each neuron can --in principle-- target any other neuron 
    (in contrast to locally coupled systems, such as crystals)
  \item<6-> sparse connectivity: at scale of $1\,\text{mm}^3$, 
    only $10\%$ of all possible connections are realized
  \item<7-> structural ``non-random'' connectivity motifs
  \item<8-> cell-, layer-, and area specific connectivity
  \item<9-> (activity dependent) modulation of connection strength (plasticity) 
    on various time scales: $\text{ms},\ldots, \text{days}$
  \item<10-> interaction between neurons (predominantly) via short ($\sim$ $2\text{ms}$) electric pulses (``action potentials'', ``spikes'') with stereotype shape (only spike time matters)
  \item<11-> delayed interaction: $\sim{}0.1,\ldots,10\,\text{ms}$
  \item<12-> high degree of spatio-temporal sparsity:\\
    at any ``moment'' in time, only a small fraction of neurons fires
  \item<13-> energy demands of the brain:
    \begin{itemize}
    \item $\approx{}20\%$ of the total body budget ($\approx{}10\times$ more expensive than muscle tissue)
    \item $\approx{}20$ Watts
    \end{itemize}
  \end{itemize}
\end{frame}
%%%%%%%%%%%%%%%%%%%%%%%%%%%%%%%%%%%%%%%%%%%%%%%%%%%%%%%%%%%%%%%%%%%%%%%%%%%%%%%%%%%%%%%%%%%%%%%%%
\begin{frame}[t,plain,allowframebreaks]
  \begin{center}
    \vspace*{\fill}
    \LARGE\emph{\it Thanks}
    \vspace*{\fill}
  \end{center}
\end{frame}
%%%%%%%%%%%%%%%%%%%%%%%%%%%%%%%%%%%%%%%%%%%%%%%%%%%%%%%%%%%%%%%%%%%%%%%%%%%%%%%%%%%%%%%%%%%%%%%%%
%% references
\setbeamertemplate{bibliography item}{}  %% remove document icon
\begin{frame}[t,plain,allowframebreaks]  
  \frametitle{References}
  \bibitemsep1ex
  \renewcommand{\bibfont}{\normalfont\small}
  \printbibliography
\end{frame}
%%%%%%%%%%%%%%%%%%%%%%%%%%%%%%%%%%%%%%%%%%%%%%%%%%%%%%%%%%%%%%%%%%%%%%%%%%%%%%%%%%%%%%%%%%%%%%%%%
\end{document}
%%%%%%%%%%%%%%%%%%%%%%%%%%%%%%%%%%%%%%%%%%%%%%%%%%%%%%%%%%%%%%%%%%%%%%%%%%%%%%%%%%%%%%%%%%%%%%%%%
%%%%%%%%%%%%%%%%%%%%%%%%%%%%%%%%%%%%%%%%%%%%%%%%%%%%%%%%%%%%%%%%%%%%%%%%%%%%%%%%%%%%%%%%%%%%%%%%%

%%% Local Variables:
%%% mode: latex
%%% TeX-master: t
%%% End:
